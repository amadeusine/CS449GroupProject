% Created 2019-12-06 Fri 14:28
% Intended LaTeX compiler: xelatex
\documentclass[allowframebreaks]{beamer}
\usepackage{graphicx}
\usepackage{grffile}
\usepackage{longtable}
\usepackage{wrapfig}
\usepackage{rotating}
\usepackage[normalem]{ulem}
\usepackage{amsmath}
\usepackage{textcomp}
\usepackage{amssymb}
\usepackage{capt-of}
\usepackage{hyperref}
\usepackage{fontspec}
\setmonofont{Fira Code}[
Contextuals={Alternate}
]
\usepackage{lstfiracode}
\usepackage{listings}
\usepackage{hyperref}
\usepackage{listings-rust}
\usepackage[backend=bibtex, style=numeric, autocite=superscript]{biblatex}
\bibliographystyle{acm}
\bibliography{parsers.bib}
\lstset{language=Rust,style=colouredRust, style=FiraCodeStyle,basicstyle=\ttfamily}
\usetheme[progressbar=head]{metropolis}
\author{Samuel Lim, Michael Cu, Elias Garcia}
\date{2019.12.06}
\title{Sprint \#3Presentation}
\subtitle{CS449}
\setmainfont{Fira Code}
\setbeamertemplate{section in toc}[sections numbered]
\hypersetup{
 pdfauthor={Samuel Lim, Michael Cu, Elias Garcia},
 pdftitle={Sprint \#3Presentation},
 pdfkeywords={},
 pdfsubject={},
 pdfcreator={Emacs 26.3 (Org mode 9.1.9)}, 
 pdflang={English}}
\begin{document}

\maketitle
\begin{frame}{Outline}
\tableofcontents
\end{frame}

\section{Project Summary}
\label{sec:orgb8f1dd0}
\begin{frame}[label={sec:orgcde1358}]{Top Level}
\begin{itemize}
\item Nine Men's Morris in the browser
\item Front-end delivers static HTML/CSS and handles user input
\item Back-end drives game logic
\end{itemize}
\end{frame}
\begin{frame}[label={sec:orgc77ff86}]{Front-End}
\begin{itemize}
\item Delivered statically via:
\begin{itemize}
\item Node.js sever
\item Browser events target API endpoints
\item Express.js pulls the engine
\item Use EJS for templating
\end{itemize}
\end{itemize}
\end{frame}
\begin{frame}[label={sec:org464cedc}]{Back-End}
\begin{itemize}
\item Back-end is game logic and heavy lifting
\begin{itemize}
\item Written entirely in \href{https://www.rust-lang.org/}{Rust}, a systems programming language
\item Use Neon library to interop with Node.js
\item Compile time guarantees for:
\begin{itemize}
\item Memory safety
\item Data race safety
\item Type Safety
\item FFI behavior
\end{itemize}
\item In addition to being as fast as C++!
\end{itemize}
\end{itemize}
\end{frame}
\section{Sprint 3 Goals}
\label{sec:orgb3ef0ef}
\begin{frame}[fragile,label={sec:orgf6c4e28}]{Finish Front \texttt{<->} Back Communication}
 \begin{itemize}
\item Sprint 1 left us with Rust defined types and a mocked front-end for rendering
\item Sprint 2 left us with a working back-end and front-end that could talk
\begin{itemize}
\item JS had no means of communicating to back-end, same goes for back-end to front-end.
\item Neon library allows Rust to compile into actual node module that can be used by JS
\end{itemize}
\item Needed to put together
\end{itemize}
\end{frame}
\begin{frame}[label={sec:org96a3364}]{AI}
\begin{itemize}
\item Create basic functionality of a game AI
\begin{itemize}
\item take turns
\item at least somewhat intelligently, preferably
\end{itemize}
\end{itemize}
\end{frame}
\begin{frame}[label={sec:orgd12d0ab}]{Polish Off Front-end}
\begin{itemize}
\item UI Theme
\item Board layout improvement
\end{itemize}
\end{frame}
\begin{frame}[label={sec:org17dd706}]{Get ready for Web Sockets}
\begin{itemize}
\item Separate web-server from game logic itself
\item Drive game logic based on requests/responses over API endpoints
\item Structure of project allows game to be played over web, over any device
\end{itemize}
\end{frame}
\section{Sprint 3 Results}
\label{sec:org24986a8}
\begin{frame}[label={sec:orga5fc01f}]{Issues resolved}
\begin{block}{The final issues were finished:}
\begin{itemize}
\item Proper back-end game validation
\begin{itemize}
\item Piece Placement
\item Moves
\item Attacking
\end{itemize}
\end{itemize}
\end{block}
\begin{block}{Relevant PR's:}
\begin{itemize}
\item \href{https://github.com/amadeusine/CS449GroupProject/pull/42}{\#42 Game Logic, Part 1}
\href{https://github.com/amadeusine/CS449GroupProject/issues/43}{\#43 Change Poll API}
\item \href{https://github.com/amadeusine/CS449GroupProject/pull/45}{\#45 Game Logic, Part 2}
\item \href{https://github.com/amadeusine/CS449GroupProject/pull/46}{\#46 AI Logic}
\end{itemize}
\end{block}
\end{frame}
\begin{frame}[fragile,allowframebreaks, label=]{Front \texttt{<->} Back Done}
 \begin{block}{API finalized}
\begin{itemize}
\item \texttt{Manager} type makes the API over FFI
\begin{itemize}
\item \texttt{poll} the back-end with the \texttt{options} and \texttt{type} of the move being made
\item checks whether is correct, or not, and returns result
\end{itemize}
\item Internally, \texttt{Manager} engages with \texttt{GameState} and \texttt{GameOpt}, which are the major \alert{internal} API
interfaces of the back-end.
\begin{itemize}
\item Only these entities have direct access to things like \texttt{Board}, \texttt{Coord}, \texttt{PositionStatus}, and
other primitive types.
\end{itemize}
\end{itemize}
\end{block}
\begin{block}{Types cross FFI}
\begin{itemize}
\item Successfully convert from Rust \texttt{->} JS and JS \texttt{->} Rust
\begin{itemize}
\item neon provides a \texttt{define\_types!} macro that provides convenient syntactic sugar for defining
what gets publicly exposed to Node.
\item \alert{everything else defined stays private to node!}
\end{itemize}
\end{itemize}
\end{block}
\begin{block}{Integration Tests}
\begin{itemize}
\item Use Mocha and Chai within Node to pull in and check the exports and logic provided by the
compiled Rust crate.
\begin{itemize}
\item Construct mock values, pass to generated rust code, and check functionality.
\end{itemize}
\item \alert{Isolates the testing logic} between what the front-end needs to worry about vs what the
back-end need to worry about.
\begin{itemize}
\item Runs separate from the unit tests internal to rust module!
\item Truly separated concerns and modulation of program logic.
\end{itemize}
\end{itemize}
\end{block}
\end{frame}
\begin{frame}[label={sec:org71c50b8}]{Front-end finished}
\begin{itemize}
\item Add endpoints on `dev-express-js` to specialise client-server communication
\item Browser has board receptors that work both on Web Socket and direct server messages
\item Separating concerns between client and server, where we can now style and switch out game logic
accordingly
\item Integration testing from above (Mocha/Chai) is now being extended to serving logic
\end{itemize}
\end{frame}
\begin{frame}[label={sec:orgba55666}]{Ready for Web Sockets}
\begin{itemize}
\item WS (game server established) -> Node (web server dispatcher)
\item we can extend the game to online multiplayer
\item Browser/server supports this, clients can play against one another
\item Where the three-tier separation scales
\begin{itemize}
\item Faster computation on native module can push directly to WS
\item Clients can interact with each other's GUI without nasty artifacting from server-server talkback
\end{itemize}
\end{itemize}
\end{frame}
\section{Sprint 3 Lessons}
\label{sec:orge424a50}
\begin{frame}[label={sec:org18db658}]{Planning time better}
\begin{itemize}
\item Kind of a toss up because of other commitments and classes
\item Particularly with sprint 3, brutal for some of us
\end{itemize}
\end{frame}
\begin{frame}[label={sec:org702f4ba}]{Documenting Changes}
\begin{itemize}
\item Didn't write sprint 3 writeup artifact as development occurred, rushed all at the end.
\item Would be a lot smarter next time to writeup as we develop.
\begin{itemize}
\item Lesson was not learned from sprint 2
\item said same thing last time
\end{itemize}
\end{itemize}
\end{frame}
\begin{frame}[label={sec:orgf6c04ee}]{Massive churn can still happen}
\begin{itemize}
\item Back-end saw probably around \textasciitilde{}1.5K LOC change over sprint 3
\begin{itemize}
\item I am not happy
\item I am not well rested
\item I am not chill
\end{itemize}
\end{itemize}
\end{frame}
\section{Future Improvements}
\label{sec:orgc94ef9d}
\begin{frame}[label={sec:orgf9f68af}]{Optimizations}
\begin{itemize}
\item Lots of low hanging fruit:
\begin{itemize}
\item back-end: excessive heap allocations
\item front-end: more player options, back-menu
\end{itemize}
\end{itemize}
\end{frame}
\begin{frame}[label={sec:orga56bf2a}]{Game AI}
\begin{itemize}
\item Behavior
\begin{itemize}
\item simple search currently
\item search state space for better moves in future
\end{itemize}
\end{itemize}
\end{frame}
\begin{frame}[label={sec:org4e7752f}]{Implement Web Sockets}
\begin{itemize}
\item Major stretch goal: play across browsers.
\item Need to figure out server hosting and communication between possibly multi-threaded processes
\item Have all infrastructure in place between rust and node
\end{itemize}
\end{frame}
\end{document}